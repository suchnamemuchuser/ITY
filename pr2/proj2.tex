\documentclass[twocolumn, a4paper, 11pt]{article}
\usepackage[czech]{babel}
\usepackage[total={18.6cm, 26cm}, centering]{geometry}
\usepackage[T1]{fontenc}
\usepackage[utf8]{inputenc}
\usepackage{lmodern}
\usepackage{amsfonts}
\usepackage{amsmath}
\usepackage{amsthm}

\newtheorem{vet}{Věta}
\newtheorem{defn}{Definice}

\begin{document}
\begin{titlepage}
    \onecolumn
    \pagenumbering{gobble}
    
    \begin{center}
        \Huge
        \textsc{Vysoké učení technické v Brně}\\
        \vspace{0.5em}
        {\fontsize{21pt}{24pt}\selectfont
        \textsc{Fakulta informačních technologií\\}}
        \vspace{0.382\textheight}
        {\fontsize{21pt}{24pt}\selectfont
        Typografie a publikování -- 2. projekt\\
        \vspace{0.6em}
        Sazba dokumentů a matematických výrazů}
        \vfill
        {\LARGE 2025 \hfill Josef Michal (xmicha84)}
    \end{center}
    
\end{titlepage}
\twocolumn
\pagenumbering{arabic}

\section*{Úvod} \label{introduction}
V~této úloze vysázíme titulní stranu a~ukázku matematického textu, v~němž se vyskytují například rovnice\,\eqref{integrals} na straně~\pageref{integrals}, Věta\,\ref{veta1} nebo Definice\,\ref{defn1}.
Pro vytvoření těchto odkazů používáme kombinace příkazů \verb|\label|, \verb|\ref|, \verb|\eqref| a \verb|\pageref|.
Před odkazy patří nezlomitelná mezera. Text zvýrazníme pomocí příkazu \verb|\emph|, 
strojopisné písmo pomocí \verb|\texttt|.
Pro \LaTeX ové příkazy (s~obráceným lomítkem) použijeme \verb|\verb|.
Titulní strana je vysázena prostředím titlepage a~nadpis je v~optickém středu
s~využitím zlatého řezu, který byl probrán na přednášce.
Na titulní straně jsou tři různé velikosti písma a~mezi dvojicemi řádků textu
je řádkování se zadanou  velikostí 0,5\,em a~0,6\,em\footnote{Použijte správnou velikost mezery mezi číslem a~jednotkou.}.
\section{Matematický text} \label{mathtext}
Symboly číselných množin sázíme makrem \verb|\mathbb|,
kaligrafická písmena  makrem \verb|\mathcal|.
Pozor na tvar i~sklon řeckých písmen: srovnejte \verb|\rho| a \verb|\varrho|.
Konstrukce \verb|${}$| nebo \verb|\mbox{}| zabrání zalomení výrazu.
Pro definice a~věty slouží prostředí definovaná příkazem \verb|\newtheorem| z~balíku amsthm.
Tato prostředí obracejí význam \verb|\emph|:
uvnitř textu sázeného kurzívou se zvýrazňuje písmem v~základním řezu.
Důkazy se někdy ukončují značkou \verb|\qed|.
\subsection{Pseudometrický prostor} \label{pseudometric}
Pro zarovnání rovností a~nerovnosti pod sebe použijte vhodné prostředí.
\begin{defn} \label{defn1}
V~pseudometrickém prostoru ${\mathcal{M} = (M, \varrho)}$ značí $M$ množinu bodů,
${\varrho : M \times M \rightarrow \mathbb{R}}$ je zobrazení zvané \emph{pseudometrika}, které pro každé body ${x,y,z \in M}$
splňuje následující podmínky:
\setcounter{equation}{0}
\begin{eqnarray}
    \varrho(x, x) & = & 0 \label{eq:1} \\
    \varrho(x, y) & = & \varrho(y, x) \label{eq:2} \\
    \varrho(x, y) + \varrho(y, z) & \geq & \varrho(x, z) \label{eq:3}
\end{eqnarray}
\end{defn}
\subsection{Metrika} \label{metric}
Funkční hodnota pseudometriky $\varrho$ se nazývá vzdálenost.
Vzdálenost každých dvou bodů je nezáporná.
\begin{vet} \label{veta1}
Pro každé dva body ${x,y \in M}$ pseudometrického prostoru ${(M, \varrho)}$ platí ${\varrho(x,y) \geq 0}$.
\end{vet}
Důkaz: Nechť $x,y \in M$ a~označme $d = \varrho(x,y)$. Využitím\,\eqref{eq:2} máme $2d = \varrho(x, y) + \varrho(y, x)$, z~nerovnosti\,\eqref{eq:3} vyplývá $2d \geq \varrho(x, x)$ a~z~rovnosti\,\eqref{eq:1} dostaneme $2d \geq \varrho(x, x) = 0$. Odtud plyne $d \geq 0$.
Zvláštním případem pseudometrických prostorů jsou prostory metrické, v~nichž dva různé body mají vždy kladnou vzdálenost.
\begin{defn} \label{defn2}
Nechť $\mathcal{M} = (M, \varrho)$ je pseudometrický prostor, v~němž platí $\varrho(x, y) > 0$, kdykoliv $x\ne y$. Potom $\mathcal{M}$ se nazývá metrický prostor a~$\varrho$ je jeho \emph{metrika}.
\end{defn}
\section{Rovnice} \label{equations}
Velikost závorek a~svislých čar je potřeba přizpůsobit jejich obsahu.
K~tomu jsou určeny modifikátory \verb|\left| a \verb|\right|.
\begin{eqnarray}
    \lim_{p\to0} \left( \frac{1}{n} \sum\limits _{i=1}^n x_i^p \right)^\frac{1}{p} = \left( \prod\limits _{i=1}^n x_i\right)^\frac{1}{n}
\end{eqnarray}
Zde vidíme, jak se vysází proměnná určující limitu v~běžném textu: $\lim_{m\to\infty} f(m)$.
Podobně je to i~s~dalšími symboly jako ${\bigcup_{N \in \mathcal{M}} N}$ či $\sum_{n=1}^{m} x^2_i$.
S~vynucením méně úsporné sazby příkazem \verb|\limits| budou vzorce vysázeny v~podobě $\lim\limits_{m\to\infty} f(m)$ a~$\sum\limits_{n=1}^{m} x^2_i$.
Složitější matematické formule sázíme mimo plynulý text pomocí prostředí \texttt{displaymath}.
\begin{eqnarray}
    {\lim\limits_{n\to\infty}\left(1+\frac{x}{n}\right)^n} & = & \sum\limits_{n=0}^{\infty}\frac{x^n}{n!}\\
    \sum\limits_{\emptyset\neq X\subseteq P}(-1)^{|X|-1}\left|\bigcap X\right| & = & \left|\bigcup P\right|\\
    -{\int_{a}^{b} f(x) \,\mathrm{d}x} & = & \int_{b}^{a} f(y) \,\mathrm{d}y \label{integrals}
\end{eqnarray}
Nezapomeňte rovnice, na které se odkazujete, označit vhodným jménem pomocí \verb|\label|.
\section{Matice} \label{matrixes}
Pro sázení matic se používá prostředí array a~závorky s výškou nastavenou pomocí
\verb|\left|, \verb|\right|.
\[
D = 
\left| 
\begin{array}{cccc}
    a_{11} & a_{12} & \dots & a_{1n} \\
    a_{21} & a_{22} & \dots & a_{2n} \\
    \vdots & \vdots & \ddots & \vdots \\
    a_{m1} & a_{m2} & \dots & a_{mn}
\end{array} 
\right|
=
\left| 
\begin{array}{cc}
    x & y \\
    t & w
\end{array} 
\right|
= xw - yt
\]
Prostředí \verb|array| lze úspěšně využít i~jinde,
například na pravé straně následující definiční rovnosti.
\[
B_{n} = 
\left\{
\begin{array}{ll}
    1 & \text{pro } n = 0 \\
    \sum\limits_{k=0}^{n-1} \binom{n}{k} B_{k} & \text{pro } n \geq 1
\end{array}
\right.
\]
Jestliže sázíme jen levou složenou závorku, pak za párovým \verb|\right|
místo závorky píšeme tečku.
\end{document}
